\documentclass[11pt]{article}
\usepackage[T1]{fontenc}
\usepackage{lmodern}

\usepackage[letterpaper,margin=1in]{geometry}

\usepackage{amsmath}
\usepackage{booktabs}
\usepackage{calc}
\usepackage{caption}
\usepackage[flushmargin,hang]{footmisc}
\usepackage{float}
\usepackage{microtype}
\usepackage{newverbs}
\usepackage{tablefootnote}
\usepackage{tabularx}
\usepackage{todonotes}
\usepackage[hyperfootnotes=false]{hyperref} % doesn't work in tabulars as currently set
\usepackage[nohyperlinks]{acronym}
\usepackage{footnotehyper}
\usepackage[strict]{changepage}
\usepackage[binary-units=true]{siunitx}
\usepackage{enumitem}
\usepackage{stackengine}


\documentclass[11pt]{article}
\usepackage[T1]{fontenc}
\usepackage{lmodern}

\usepackage[letterpaper,margin=1in]{geometry}

\usepackage{amsmath}
\usepackage{booktabs}
\usepackage{calc}
\usepackage{caption}
\usepackage[flushmargin,hang]{footmisc}
\usepackage{float}
\usepackage{microtype}
\usepackage{newverbs}
\usepackage{tablefootnote}
\usepackage{tabularx}
\usepackage{todonotes}
\usepackage[hyperfootnotes=false]{hyperref} % doesn't work in tabulars as currently set
\usepackage[nohyperlinks]{acronym}
\usepackage{footnotehyper}
\usepackage[strict]{changepage}
\usepackage[binary-units=true]{siunitx}
\usepackage{enumitem}
\usepackage{stackengine}


\documentclass[11pt]{article}
\usepackage[T1]{fontenc}
\usepackage{lmodern}

\usepackage[letterpaper,margin=1in]{geometry}

\usepackage{amsmath}
\usepackage{booktabs}
\usepackage{calc}
\usepackage{caption}
\usepackage[flushmargin,hang]{footmisc}
\usepackage{float}
\usepackage{microtype}
\usepackage{newverbs}
\usepackage{tablefootnote}
\usepackage{tabularx}
\usepackage{todonotes}
\usepackage[hyperfootnotes=false]{hyperref} % doesn't work in tabulars as currently set
\usepackage[nohyperlinks]{acronym}
\usepackage{footnotehyper}
\usepackage[strict]{changepage}
\usepackage[binary-units=true]{siunitx}
\usepackage{enumitem}
\usepackage{stackengine}


\documentclass[11pt]{article}
\usepackage[T1]{fontenc}
\usepackage{lmodern}

\usepackage[letterpaper,margin=1in]{geometry}

\usepackage{amsmath}
\usepackage{booktabs}
\usepackage{calc}
\usepackage{caption}
\usepackage[flushmargin,hang]{footmisc}
\usepackage{float}
\usepackage{microtype}
\usepackage{newverbs}
\usepackage{tablefootnote}
\usepackage{tabularx}
\usepackage{todonotes}
\usepackage[hyperfootnotes=false]{hyperref} % doesn't work in tabulars as currently set
\usepackage[nohyperlinks]{acronym}
\usepackage{footnotehyper}
\usepackage[strict]{changepage}
\usepackage[binary-units=true]{siunitx}
\usepackage{enumitem}
\usepackage{stackengine}


\input{bedRModv2.ver}

\hypersetup{colorlinks=true,
  linkcolor=blue,
  filecolor=magenta,
  urlcolor=blue,
  pdfinfo={githash=\commitdesc}}

\definecolor{cverbbg}{gray}{0.93}

\title{The \acf{bedRMod} format}
\author{Transregio 319 RMaP}
\date{\headdate}

\setlength{\emergencystretch}{\hsize}
\setlength{\footnotemargin}{1em}

\floatplacement{table}{htbp}
\setcounter{topnumber}{2}
\setcounter{bottomnumber}{2}
\setcounter{totalnumber}{4}
\setcounter{dbltopnumber}{2}
\renewcommand{\dbltopfraction}{0.9}
\renewcommand{\textfraction}{0.07}
\renewcommand{\floatpagefraction}{0.7}

\interfootnotelinepenalty=1000000
\makesavenoteenv{tabularx}

\newcolumntype{L}{>{\raggedright\arraybackslash}X}

\providecommand*{\Ac}[1]{\ac{#1}} % work around outdated acronym.sty packages
\newcommand*{\acrodefused}[2]{\acrodef{#1}{#2}\acused{#1}}

\frenchspacing

% eliminate passive voice warnings
% chktex-file -3

\begin{document}

\maketitle

\begin{small}
\noindent
The master version of this document can be found at \url{https://github.com/dieterich-lab/euf-specs}.
This printing is version~\commitdesc\ from that repository, last modified on the date shown above.
\end{small}

\acused{ASCII}

\section{Specification}

\Ac{bedRMod} is a tab-delimited~\textbf{file} format, compatible with the \acf{BED} format\footnote{SAM/BAM and related specifications, \url{http://samtools.github.io/hts-specs}}. Metadata are in~\textbf{header line}s, which describe metainformation about the source of the data. Data are in~\textbf{data line}s, which describe \emph{RNA modification}s, or sites of putative \emph{RNA modification}s, by physical start and end position on a linear chromosome. The metadata must be consistent for all~\textbf{data line}s. The~\textbf{file} extension for the \ac{bedRMod} format is~\texttt{.bedrmod},~\texttt{.bedmethyl}, or~\texttt{.bed}.

\subsection{Scope}

This specification is a variation of the \ac{BED} description for~\textbf{data line}s. The content of this document is directly inspired from the 
official \ac{BED} specifications. Only the most important or less obvious concepts are reiterated in this document. For general information, refer to the official \ac{BED} specifications. Potential interoperability issues with the \ac{BED} format are also described in this document.

\subsection{Typographic conventions}

This document uses the official \ac{BED} typographic conventions~(\autoref{tab:typographic-conventions}).

\begin{savenotes}
  \begin{table}
    \begin{tabularx}{\textwidth}{r L L}
      \toprule
      Style & Meaning & Examples \\
      \midrule
      Bold & Terms defined in subsections~\ref{sec:terms}--\ref{sec:lines} & \textbf{file}{\quad}\textbf{line} \\
      Sans serif & Names of~\textbf{field}s & \textsf{chrom}{\quad}\textsf{chromStart}{\quad}\textsf{chromEnd} \\
      Fixed-width & Literals or \ac{regex}es\footnote{POSIX/IEEE~1003.1--2017 Extended Regular Expressions, for the ``C'' locale.
                    \emph{IEEE Standard for Information Technology---Portable Operating System Interface~(POSIX) Base Specifications}, IEEE~1003.1--2017, 2017} & \texttt{.bedrmod}{\quad}\texttt{grep}{\quad}\texttt{[[:alnum:]]+} \\
      \bottomrule
    \end{tabularx}
    \caption{\textbf{Typographic conventions.}}\label{tab:typographic-conventions}
  \end{table}
\end{savenotes}

\subsection{Terminology and concepts}\label{sec:terms}
\begin{description}
\item[0-based, half-open coordinate system:]
  A coordinate system where the first base starts at position~0, and the start of the interval is included but the end is not.
  For example, for a sequence of bases~\texttt{ACTGCG}, the bases given by the interval~[2,~4) are~\texttt{TG}. % chktex 9

\item[\acs{bedRMod} field:]
  One of the 11~standard~\textbf{field}s defined in this specification.
  All~\textbf{\acs{bedRMod} field}s are mandatory.

\item[comment line:]
  A~\textbf{line} that starts with~\texttt{\#} with no horizontal whitespace beforehand. \textbf{Comment line}s at the start of 
  the~\textbf{file} are~\textbf{header line}s defined in this specification.

\item[custom field:]
  A~\textbf{field} defined by the~\textbf{file}~creator.
  \textbf{Custom field}s occur in each~\textbf{line} after any~\textbf{\acs{bedRMod} field}s.

\item[data line:]
  A~\textbf{line} that contains~\textbf{feature}~data.

\item[feature:]
  A linear region of a chromosome reporting the presence or absence of a given RNA modification, supported by quantitative evidence, typically at single-base resolution, but can include a context.

\item[field:]
  Data stored as non-tab text.
  All~\textbf{field}s are 7-bit US \ac{ASCII} printable characters\footnote{Characters in the range~\texttt{{\textbackslash}x20} to~\texttt{{\textbackslash}x7e}, therefore not including any control characters}.

\item[field separator:]
  One or more horizontal whitespace characters (space or tab).
  The~\textbf{field separator} must match the \ac{regex}~\texttt{[ {\textbackslash}t]+}.
  This specification strongly recommends using tab as~\textbf{field separator} throughout the~\textbf{file}.

\item[file:]
  Sequence of one or more~\textbf{data line}s with a~\textbf{header}.

\item[header:]
  Mandatory~\textbf{header line}s, followed by optional~\textbf{comment line}s, at the start of the~\textbf{file}.

\item[header field:]
  A mandatory~\emph{key=value} pair describing one of the~\textbf{header line}s. 
  
\item[header line:]
  A~\textbf{line} that starts with~\texttt{\#} with no horizontal whitespace beforehand, immediately followed by a~\textbf{header field}.
  
\item[line:]
  String terminated by a~\textbf{line separator}, in one of the following classes.
  Either a~\textbf{data line} or a~\textbf{comment line}, \textit{cf.}~\autoref{sec:lines}.

\item[line separator:]
  Either carriage return~(\texttt{{\textbackslash}r}, equivalent to~\texttt{{\textbackslash}x0d}), newline~(\texttt{{\textbackslash}n}, equivalent to~\texttt{{\textbackslash}x0a}), or carriage return followed by newline~(\texttt{{\textbackslash}r{\textbackslash}n}, equivalent to~\texttt{{\textbackslash}x0d{\textbackslash}x0a}).
  The same~\textbf{line separator} must be used throughout the~\textbf{file}.
\end{description}

\subsection{Lines}\label{sec:lines}

\subsubsection{Data lines}

\textbf{Data line}s contain~\textbf{feature}~data (RNA modification).
A~\textbf{data line} is composed of~\textbf{field}s separated by~\textbf{field separator}s.

\subsubsection{Comment lines}

\textbf{Comment line}s provide no~\textbf{feature} data. They start with~\texttt{\#} with no horizontal whitespace beforehand.
\textbf{Comment line}s at the beginning of the~\textbf{file} are treated as~\textbf{header line}s, and must conform to~\textbf{header} specifications, \textit{cf.}~\autoref{sec:header}. A~\texttt{\#} appearing anywhere else in a~\textbf{data line} is treated as~\textbf{feature} data, not a comment.


\subsection{Header specification}\label{sec:header}

The~\textbf{header} contains metainformation about the source of the data. Each~\textbf{header line} starts with a~\texttt{\#} and contains a
mandatory~\textbf{header field} in the form of a~\emph{key=value} pair, \textit{e.g.}~\texttt{\#fileformat=bedRModv2}~(\autoref{tab:header}). All~\textbf{header field}s are mandatory. The first seven~\textbf{header field}s must be assigned a value, and the value must generally follow a controlled vocabulary; the remaining~\textbf{header field}s are free text, and can be left without a value, although it is strongly advised to provide a value for each one. Additional~\textbf{line}s starting with~\texttt{\#} are treated as~\textbf{comment line}s.

A \ac{bedRMod}~\textbf{header} describes information for one organism, one assembly and annotation, and one modification (RNA) type, hence a \ac{bedRMod}~\textbf{file} contains~\textbf{data lines} for one organism, one assembly and annotation, and one modification (RNA) type. A \ac{bedRMod}~\textbf{file} can contain~\textbf{data lines} for different RNA modifications, \textit{e.g.}~\texttt{m6A} and~\texttt{m5C}, \textit{cf.}~\autoref{sec:data}.


\begin{savenotes}
  \begin{table}
    \begin{tabularx}{\textwidth}{X p{.5\textwidth} p{.15\textwidth}}
      \toprule
      Header Field & Brief description & Value required \\
      \midrule
      \textsf{fileformat} & Fileformat and version \textit{e.g.}~\texttt{bedRModv2} & Yes \\
      \textsf{organism} & NCBI Taxonomic identifier\footnote{NCBI Taxonomy: a comprehensive update on curation, resources and tools, \url{10.1093/database/baaa062}} & Yes \\
      \textsf{modification\textunderscore type} & RNA & Yes \\
      \textsf{modification\textunderscore names} & name:MODOMICS \emph{short name}:reference or unmodified base \textit{e.g.}~\texttt{21891:m6A:A} & Yes \\
      \textsf{assembly} & Genome or transcriptome assembly \textit{e.g.}~\texttt{GRCh38} & Yes \\
      \textsf{annotation\textunderscore source} & Annotation source \textit{e.g.}~\texttt{Ensembl} & Yes \\
      \textsf{annotation\textunderscore version} & Annotation version \textit{e.g.}~\texttt{110} & Yes \\
      \textsf{sequencing\textunderscore platform} & Sequencing platform \textit{e.g.}~\texttt{Illumina NovaSeq 6000}, or~\texttt{ONT MinION} & No \\
      \textsf{basecalling} & Basecalling model information where relevant & No \\
      \textsf{bioinformatics\textunderscore workflow} & Link to bioinformatics workflow, program name, version, and/or call, or information relevant to score, coverage, or frequency calculation, \textit{etc.} & No \\
      \textsf{experiment} & Information about experimental protocol and design, or link to \textit{e.g.} openBIS or openAIRE repository & No \\
      \textsf{external\textunderscore source} & Databank:ID of data \textit{e.g.}~\texttt{GEO:GSEXXXXXX} & No \\
      \bottomrule
    \end{tabularx}
    \caption{\textbf{Header Fields.}}\label{tab:header}
  \end{table}
\end{savenotes}
 

\subsection{Data specification}\label{sec:data}

Each~\textbf{data line} contains 11~\textbf{\acs{bedRMod} field}s delimited by a~\textbf{field separator} (tab).
All~\textbf{fields} are mandatory~(\autoref{tab:fields}). Missing data is not allowed.
Additional optional~\textbf{field}s can be added, following the first 11~\textbf{field}s, according to the \acs{BED} specifications, but 
it is not recommended to use \acs{bedRMod} with exactly 12~\textbf{field}s, \textit{cf.}~\autoref{sec:custom_fields}.

\begin{savenotes}
  \begin{table}
    \begin{adjustwidth}{-0.5in}{-0.5in}
      \begin{tabularx}{\linewidth}{r l l l L}
        \toprule
        Col & \acs{bedRMod} Field & Type & Regex or range & Brief description \\
        \midrule
        1
        & \textsf{chrom}
        & String
        & \texttt{[[:alnum:]\_]\{1,255\}}\footnote{\texttt{[[:alnum:]\_]} is equivalent to the \ac{regex}~\texttt{[A-Za-z0-9\_]}. % chktex 8
        It is also equivalent to the Perl extension~\texttt{[[:word:]]}}
        & Chromosome name \\
        2 & \textsf{chromStart} & Int & $[0, 2^{64}-1]$ & \textbf{Feature} start position \\
        3 & \textsf{chromEnd} & Int & $[0, 2^{64} -1]$ & \textbf{Feature} end position \\
        4 
        & \textsf{name} 
        & String 
        & \texttt{[{\textbackslash}x20-{\textbackslash}x7e]\{1,255\}} 
        & Modification name and additional attributes \\
        5 & \textsf{score} & String & \texttt{[{\textbackslash}x20-{\textbackslash}x7e]\{1,255\}} & Modification confidence \\
        6 & \textsf{strand} & String & \texttt{[-+.]} & \textbf{Feature} strand \\
        7 & \textsf{thickStart} & Int & $[0, 2^{64}-1]$ & Thick start position, typically same as~\textsf{chromStart} \\
        8 & \textsf{thickEnd} & Int & $[0, 2^{64}-1]$ & Thick end position, typically same as~\textsf{chromEnd} \\
        9 & \textsf{itemRgb} & Int,Int,Int & \texttt{(}$[0, 255], [0,255], [0,255]$\texttt{) | 0} & Display color \\ % chktex 9
        10 & \textsf{coverage} & Int &  $(0, 2^{64}-1]$ & Coverage \\
        11 & \textsf{frequency} & Float & $[0, 100]$ & Percentage of modification \\
        \bottomrule
      \end{tabularx}
    \end{adjustwidth}
    \caption{\textbf{\acs{bedRMod} Fields.}}\label{tab:fields}
  \end{table}
\end{savenotes}

In a \ac{bedRMod}~\textbf{file}, each~\textbf{data line} must have the same number of~\textbf{field}s.
The positions in~\textbf{\acs{bedRMod} field}s are all described in the~\textbf{0-based, half-open coordinate system}, exactly as 
described in the official \ac{BED} specifications.

\subsection{Coordinates}
Refer to the official \ac{BED} specifications.

\subsection{Simple attributes}
\begin{enumerate}
\item \textsf{name}: String that describes the~\textbf{feature}, \textit{i.e.} the modification. \textsf{name} must describe 
the modification using the \emph{short name} of the MODOMICS nomenclature\footnote{MODOMICS, \url{https://www.genesilico.pl/modomics/modifications}}, or the base modification code described in the SAMtags\footnote{SAM tags, \url{https://samtools.github.io/hts-specs}}, or a numeric ChEBI code\footnote{Chemical Entities of Biological Interest, \url{https://www.ebi.ac.uk/chebi}}. The MODOMICS \emph{short name} corresponding to~\textsf{name} is always described in the~\textbf{header field}~\textsf{modification\textunderscore names}. Additional name attributes are allowed, and must be comma-separated \textit{e.g.}~\texttt{a,DRACH,2}.

\item \textsf{score}: String representation of the confidence in calling this modification. Any measure of confidence is valid, but a \ac{bedRMod}~\textbf{file} with non-integer-like~\textsf{score} values outside the range $[0, 1000]$ may fail from being correctly displayed in a visual representation\footnote{\textit{cf.} bedtools definition of score, \url{https://bedtools.readthedocs.io/en/latest/content/general-usage.html?highlight=bed\%20format}}.

\item \textsf{coverage}: Integer between~0 and the maximum size of an unsigned 64-bit integer, excluding~0, representing the number of reads aligned at this position or the number of calls for this~\textbf{feature}.

\item \textsf{frequency}: Float\footnote{Decimal string representation of 64-bit floating point number, IEEE Standard for Binary Floating-Point Arithmetic. IEEE 754–1985, 1985.} between~0 and~100, including~0, representing the modification frequency, or stoichiometry. This can be the percentage of modified reads, or the ratio of the number of calls passing filters that were classified as a residue with the base modification reported for this~\textbf{feature} to the \emph{valid coverage}, multiplied by~100. See~\autoref{sec:fields} for an explanation of \emph{valid coverage}. A~\textsf{frequency} of~0 means that there is evidence that a given site is not modified, \textit{i.e.} the canonical/unmodified base is reported with a confidence quantified by~\textsf{score}~$>$0.
\end{enumerate}

\subsection{Display attributes}
\begin{enumerate}
  \setcounter{enumi}{4}

\item \textsf{thickStart}: Included for compatibility, typically same as~\textsf{chromStart}.

\item \textsf{thickEnd}: Included for compatibility, typically same as~\textsf{chromEnd}.

\item \textsf{itemRgb}: Included for compatibility, typically~\texttt{0,0,0}.

\end{enumerate}

\subsection{Custom fields}\label{sec:custom_fields}

\textbf{Custom field}s defined by the~\textbf{file} creator may contain any printable 7-bit US \ac{ASCII} character (which includes spaces, but excludes tabs, newlines, and other control characters), as defined by the \ac{BED} format definitions.

A \acs{bedRMod}~\textbf{file} with exactly 12~\textbf{field}s, \textit{i.e.} containing one additional optional~\textbf{field}, may be implicitely 
assumed to be a~\textbf{BED12}~\textbf{file} by certain software and genome browsers, which can result in unexpected behaviour! 

\section{Examples}

\subsection[title]{Example bedRMod file from the \acs{bedRMod} and related specifications\footnote{\url{https://github.com/dieterich-lab/euf-specs/examples/bedrmod/example.bedrmod}}}\label{sec:example-bedrmod}

\begin{verbatim}
#fileformat=bedRModv2
#organism=9606
#modification_type=RNA
#modification_names=20607:m5C:C,21891:m6A:A
#assembly=GRCh38
#annotation_source=Ensembl
#annotation_version=110
#sequencing_platform=Illumina NovaSeq 6000
#basecalling=
#bioinformatics_workflow=workflow:https://github.com/XXX
#experiment=https://doi.org/10.XXX
#external_source=SRA:PRJNAXXXXXX,GEO:GSEXXXXXX
#chrom	chromStart	chromEnd	name	score	strand	thickStart	thickEnd	itemRgb	coverage	frequency
1	1391918	1391919	20607	0	-	1391918	1391919	0,0,0	42	42.56
2	8878712	8878713	20607	0	-	8878712	8878713	0,0,0	318	44.23
3	11980442	11980443	21891	0	+	11980442	11980443	0,0,0	111	56.20
4	17054111	17054112	20607	0	-	17054111	17054112	0,0,0	40	34.03
5	23691799	23691800	21891	0	+	23691799	23691800	0,0,0	352	27.33
\end{verbatim}

\section{Recommended practice for the \acs{bedRMod} format}

\subsection{\acs{bedRMod} extension}
The~\textbf{file} extension is~\texttt{.bedrmod},~\texttt{.bedmethyl}, or~\texttt{.bed}. Since the \ac{BED} format prohibits~\textbf{BED11}, there should be little confusion in general, but when~\textbf{custom field}s are defined, it is recommended to use the~\texttt{.bedrmod} or~\texttt{.bedmethyl} extension. \Ac{bedRMod} formalizes the ENCORE bedMethyl format\footnote{Description of bedMethyl file, \url{https://www.encodeproject.org/data-standards/wgbs}} for \emph{RNA modification}s.

\subsection{Mandatory \acs{bedRMod} header fields}
These~\textbf{header field}s are not free text, and must conform to a controlled vocabulary.
\begin{itemize}
\item \textsf{fileformat}: A valid version of this specification, including the format name, \textit{e.g.}~\texttt{bedRModv2}. 

\item \textsf{modification\textunderscore type}: A valid RNA type\footnote{For example RNA or mRNA, tRNA, or rRNA. A next version of this specification should prescribe a controlled vocabulary by providing a reference to an established RNA ontology.}.

\item \textsf{modification\textunderscore names}: A comma-separated dictionary mapping in the form name:MODOMICS \emph{short name}:reference or unmodified base, where name corresponds to the~\textsf{name}~\textbf{field} of a~\textbf{data line}, MODOMICS \emph{short name} to the corresponding short name of the MODOMICS nomenclature, and reference or unmodified base is the unmodified base as reported by the sequencing instrument for the top strand, as described in the SAMtags, except that `N' is not allowed (since it can be used to match any base). A value is also required when using MODOMICS \emph{short names}, \textit{e.g.}~\texttt{m6A:m6A:A}. All modifications present in the~\textbf{file} must be included, in a comma-separated list of items.

\item \textsf{organism}: A valid NCBI Taxonomic identifier, \textit{e.g.}~\texttt{9606}.

\item \textsf{assembly}: The name of a valid assembly, \textit{e.g.} using the Ensembl terminology,~\texttt{GRCh38}. 

\item \textsf{annotation\textunderscore source}: The name of a valid annotation, \textit{e.g.}~\texttt{Ensembl}. 

\item \textsf{annotation\textunderscore version}: A valid version for the annotation source, \textit{e.g.}~\texttt{110}. 
\end{itemize}

\subsection{\acs{bedRMod} header fields}
These~\textbf{header field}s can be left without a value, but the key must always be present. The value is free text, although it is strongly recommended
to reference established \textsf{sequencing\textunderscore platform}s, \textsf{basecalling} models, \textsf{bioinformatics\textunderscore workflow}s, or \textsf{external\textunderscore source}s using an exact terminology and/or recognized identifiers.  
\begin{itemize}
\item \textsf{sequencing\textunderscore platform}: Typically, the name of the sequencing instrument or device, including key specifications if relevant, \textit{e.g.}~\texttt{ONT MinION}.

\item \textsf{basecalling}: Basecalling model such as name of versioned model, reference to published model, and/or additional details on training, \textit{e.g.}~\texttt{dna\textunderscore r9.4.1\textunderscore e8\textunderscore sup$@$v3.6}.

\item \textsf{bioinformatics\textunderscore workflow}: Program name, version, and/or call used to generate the~\textbf{file}, or link to open source bioinformatics workflow, including version and/or any additional details to facilitate data lineage. The information should be sufficient to reproduce the content of the~\textbf{file}.

\item \textsf{experiment}: Supplementary information about experimental protocol, design, or the content of the~\textbf{file} such as conditions used, number of replicates, \textit{etc.}, or link to an openAIRE repository.

\item \textsf{external\textunderscore source}: A comma-separated list of sources of the form Databank:ID, \textit{e.g.}~~\texttt{GEO:GSEXXXXXX, Zenodo:10.XXX/zenodo.XXXXXXXX}. Free text is allowed, but this specification recommends using the format described here. Together with~\textsf{bioinformatics\textunderscore workflow}, this should allow to reproduce the content of the~\textbf{file}.
\end{itemize}


\subsection{\acs{bedRMod} fields}\label{sec:fields}
\begin{itemize}
\item \textsf{chrom}: The name of each chromosome should match the names from a reference genome assembly, as given in the~\textbf{header}.
  For example, if~\texttt{\#assembly=GRCh38}, then chromosomes should be named~\texttt{1} to~\texttt{22}, \texttt{X}, \texttt{Y}, and~\texttt{MT},
  consistently through the~\textbf{file}.
\item \textsf{name}: The MODOMICS \emph{short names} or the ChEBI codes should be used in preference to the base modification codes described in the SAMtags.
\item \textsf{score}: The \emph{valid coverage} should be used as a measure of confidence. This can be \textit{e.g.} the number of calls passing filters (classified as modified and unmodified) at the reported modification position\footnote{\textit{cf.} \url{https://nanoporetech.github.io/modkit}}, or the number or reads remaining after filtering and used to infer the modificaton status. If \emph{valid coverage} is not available, then~\textsf{coverage} can be used as a measure of confidence.
\end{itemize}

\subsection{Whitespace}\label{sec:whitespace}
We recommend that only a single tab~(\texttt{{\textbackslash}t}) be used as~\textbf{field separator}, \textit{cf.} offical \ac{BED} specifications.

\section{Information supplied out-of-band}

A \ac{bedRMod} \textbf{file} contains 11 required~\textbf{field}s, any additional~\textbf{field}s may require information that must be supplied out-of-band.
A common practice is to include a~\textbf{comment line} after the~\textbf{header} to describe the~\textbf{field}s used in the~\textbf{file}, \textit{cf.}~\autoref{sec:example-bedrmod}.

The semantics of~\textbf{field}s such as~\textsf{score},~\textsf{coverage}, and~\textsf{frequency} can be included in the~\textsf{header} using the \textbf{header field}~\texttt{bioinformatics\textunderscore workflow}.

\section{Acronyms}

% using the optional argument to acronym to set the label width causes it to use the list environment instead of description, which means we can't set nosep easily
\setlist[description]{labelwidth=\widthof{\textbf{\acs{bedRMod}}},nosep}
\begin{acronym}
  \acro{ASCII}{American Standard Code for Information Interchange}
  \acro{BED}{Browser Extensible Data}
  \acro{bedRMod}{Browser Extensible Data for RNA modification}
  \acro{ChEBI}{Chemical Entities of Biological Interest}
  \acro{NCBI}{National Center for Biotechnology Information}
  \acro{regex}{regular expression}
\end{acronym}

\section{Acknowledgments}

We thank the \acf{bedRMod} format specification working group and the Modkit developers.

\end{document}

% chktex-file 17

%%% Local Variables:
%%% mode: latex
%%% TeX-master: t
%%% End:


\hypersetup{colorlinks=true,
  linkcolor=blue,
  filecolor=magenta,
  urlcolor=blue,
  pdfinfo={githash=\commitdesc}}

\definecolor{cverbbg}{gray}{0.93}

\title{The \acf{bedRMod} format}
\author{Transregio 319 RMaP}
\date{\headdate}

\setlength{\emergencystretch}{\hsize}
\setlength{\footnotemargin}{1em}

\floatplacement{table}{htbp}
\setcounter{topnumber}{2}
\setcounter{bottomnumber}{2}
\setcounter{totalnumber}{4}
\setcounter{dbltopnumber}{2}
\renewcommand{\dbltopfraction}{0.9}
\renewcommand{\textfraction}{0.07}
\renewcommand{\floatpagefraction}{0.7}

\interfootnotelinepenalty=1000000
\makesavenoteenv{tabularx}

\newcolumntype{L}{>{\raggedright\arraybackslash}X}

\providecommand*{\Ac}[1]{\ac{#1}} % work around outdated acronym.sty packages
\newcommand*{\acrodefused}[2]{\acrodef{#1}{#2}\acused{#1}}

\frenchspacing

% eliminate passive voice warnings
% chktex-file -3

\begin{document}

\maketitle

\begin{small}
\noindent
The master version of this document can be found at \url{https://github.com/dieterich-lab/euf-specs}.
This printing is version~\commitdesc\ from that repository, last modified on the date shown above.
\end{small}

\acused{ASCII}

\section{Specification}

\Ac{bedRMod} is a tab-delimited~\textbf{file} format, compatible with the \acf{BED} format\footnote{SAM/BAM and related specifications, \url{http://samtools.github.io/hts-specs}}. Metadata are in~\textbf{header line}s, which describe metainformation about the source of the data. Data are in~\textbf{data line}s, which describe \emph{RNA modification}s, or sites of putative \emph{RNA modification}s, by physical start and end position on a linear chromosome. The metadata must be consistent for all~\textbf{data line}s. The~\textbf{file} extension for the \ac{bedRMod} format is~\texttt{.bedrmod},~\texttt{.bedmethyl}, or~\texttt{.bed}.

\subsection{Scope}

This specification is a variation of the \ac{BED} description for~\textbf{data line}s. The content of this document is directly inspired from the 
official \ac{BED} specifications. Only the most important or less obvious concepts are reiterated in this document. For general information, refer to the official \ac{BED} specifications. Potential interoperability issues with the \ac{BED} format are also described in this document.

\subsection{Typographic conventions}

This document uses the official \ac{BED} typographic conventions~(\autoref{tab:typographic-conventions}).

\begin{savenotes}
  \begin{table}
    \begin{tabularx}{\textwidth}{r L L}
      \toprule
      Style & Meaning & Examples \\
      \midrule
      Bold & Terms defined in subsections~\ref{sec:terms}--\ref{sec:lines} & \textbf{file}{\quad}\textbf{line} \\
      Sans serif & Names of~\textbf{field}s & \textsf{chrom}{\quad}\textsf{chromStart}{\quad}\textsf{chromEnd} \\
      Fixed-width & Literals or \ac{regex}es\footnote{POSIX/IEEE~1003.1--2017 Extended Regular Expressions, for the ``C'' locale.
                    \emph{IEEE Standard for Information Technology---Portable Operating System Interface~(POSIX) Base Specifications}, IEEE~1003.1--2017, 2017} & \texttt{.bedrmod}{\quad}\texttt{grep}{\quad}\texttt{[[:alnum:]]+} \\
      \bottomrule
    \end{tabularx}
    \caption{\textbf{Typographic conventions.}}\label{tab:typographic-conventions}
  \end{table}
\end{savenotes}

\subsection{Terminology and concepts}\label{sec:terms}
\begin{description}
\item[0-based, half-open coordinate system:]
  A coordinate system where the first base starts at position~0, and the start of the interval is included but the end is not.
  For example, for a sequence of bases~\texttt{ACTGCG}, the bases given by the interval~[2,~4) are~\texttt{TG}. % chktex 9

\item[\acs{bedRMod} field:]
  One of the 11~standard~\textbf{field}s defined in this specification.
  All~\textbf{\acs{bedRMod} field}s are mandatory.

\item[comment line:]
  A~\textbf{line} that starts with~\texttt{\#} with no horizontal whitespace beforehand. \textbf{Comment line}s at the start of 
  the~\textbf{file} are~\textbf{header line}s defined in this specification.

\item[custom field:]
  A~\textbf{field} defined by the~\textbf{file}~creator.
  \textbf{Custom field}s occur in each~\textbf{line} after any~\textbf{\acs{bedRMod} field}s.

\item[data line:]
  A~\textbf{line} that contains~\textbf{feature}~data.

\item[feature:]
  A linear region of a chromosome reporting the presence or absence of a given RNA modification, supported by quantitative evidence, typically at single-base resolution, but can include a context.

\item[field:]
  Data stored as non-tab text.
  All~\textbf{field}s are 7-bit US \ac{ASCII} printable characters\footnote{Characters in the range~\texttt{{\textbackslash}x20} to~\texttt{{\textbackslash}x7e}, therefore not including any control characters}.

\item[field separator:]
  One or more horizontal whitespace characters (space or tab).
  The~\textbf{field separator} must match the \ac{regex}~\texttt{[ {\textbackslash}t]+}.
  This specification strongly recommends using tab as~\textbf{field separator} throughout the~\textbf{file}.

\item[file:]
  Sequence of one or more~\textbf{data line}s with a~\textbf{header}.

\item[header:]
  Mandatory~\textbf{header line}s, followed by optional~\textbf{comment line}s, at the start of the~\textbf{file}.

\item[header field:]
  A mandatory~\emph{key=value} pair describing one of the~\textbf{header line}s. 
  
\item[header line:]
  A~\textbf{line} that starts with~\texttt{\#} with no horizontal whitespace beforehand, immediately followed by a~\textbf{header field}.
  
\item[line:]
  String terminated by a~\textbf{line separator}, in one of the following classes.
  Either a~\textbf{data line} or a~\textbf{comment line}, \textit{cf.}~\autoref{sec:lines}.

\item[line separator:]
  Either carriage return~(\texttt{{\textbackslash}r}, equivalent to~\texttt{{\textbackslash}x0d}), newline~(\texttt{{\textbackslash}n}, equivalent to~\texttt{{\textbackslash}x0a}), or carriage return followed by newline~(\texttt{{\textbackslash}r{\textbackslash}n}, equivalent to~\texttt{{\textbackslash}x0d{\textbackslash}x0a}).
  The same~\textbf{line separator} must be used throughout the~\textbf{file}.
\end{description}

\subsection{Lines}\label{sec:lines}

\subsubsection{Data lines}

\textbf{Data line}s contain~\textbf{feature}~data (RNA modification).
A~\textbf{data line} is composed of~\textbf{field}s separated by~\textbf{field separator}s.

\subsubsection{Comment lines}

\textbf{Comment line}s provide no~\textbf{feature} data. They start with~\texttt{\#} with no horizontal whitespace beforehand.
\textbf{Comment line}s at the beginning of the~\textbf{file} are treated as~\textbf{header line}s, and must conform to~\textbf{header} specifications, \textit{cf.}~\autoref{sec:header}. A~\texttt{\#} appearing anywhere else in a~\textbf{data line} is treated as~\textbf{feature} data, not a comment.


\subsection{Header specification}\label{sec:header}

The~\textbf{header} contains metainformation about the source of the data. Each~\textbf{header line} starts with a~\texttt{\#} and contains a
mandatory~\textbf{header field} in the form of a~\emph{key=value} pair, \textit{e.g.}~\texttt{\#fileformat=bedRModv2}~(\autoref{tab:header}). All~\textbf{header field}s are mandatory. The first seven~\textbf{header field}s must be assigned a value, and the value must generally follow a controlled vocabulary; the remaining~\textbf{header field}s are free text, and can be left without a value, although it is strongly advised to provide a value for each one. Additional~\textbf{line}s starting with~\texttt{\#} are treated as~\textbf{comment line}s.

A \ac{bedRMod}~\textbf{header} describes information for one organism, one assembly and annotation, and one modification (RNA) type, hence a \ac{bedRMod}~\textbf{file} contains~\textbf{data lines} for one organism, one assembly and annotation, and one modification (RNA) type. A \ac{bedRMod}~\textbf{file} can contain~\textbf{data lines} for different RNA modifications, \textit{e.g.}~\texttt{m6A} and~\texttt{m5C}, \textit{cf.}~\autoref{sec:data}.


\begin{savenotes}
  \begin{table}
    \begin{tabularx}{\textwidth}{X p{.5\textwidth} p{.15\textwidth}}
      \toprule
      Header Field & Brief description & Value required \\
      \midrule
      \textsf{fileformat} & Fileformat and version \textit{e.g.}~\texttt{bedRModv2} & Yes \\
      \textsf{organism} & NCBI Taxonomic identifier\footnote{NCBI Taxonomy: a comprehensive update on curation, resources and tools, \url{10.1093/database/baaa062}} & Yes \\
      \textsf{modification\textunderscore type} & RNA & Yes \\
      \textsf{modification\textunderscore names} & name:MODOMICS \emph{short name}:reference or unmodified base \textit{e.g.}~\texttt{21891:m6A:A} & Yes \\
      \textsf{assembly} & Genome or transcriptome assembly \textit{e.g.}~\texttt{GRCh38} & Yes \\
      \textsf{annotation\textunderscore source} & Annotation source \textit{e.g.}~\texttt{Ensembl} & Yes \\
      \textsf{annotation\textunderscore version} & Annotation version \textit{e.g.}~\texttt{110} & Yes \\
      \textsf{sequencing\textunderscore platform} & Sequencing platform \textit{e.g.}~\texttt{Illumina NovaSeq 6000}, or~\texttt{ONT MinION} & No \\
      \textsf{basecalling} & Basecalling model information where relevant & No \\
      \textsf{bioinformatics\textunderscore workflow} & Link to bioinformatics workflow, program name, version, and/or call, or information relevant to score, coverage, or frequency calculation, \textit{etc.} & No \\
      \textsf{experiment} & Information about experimental protocol and design, or link to \textit{e.g.} openBIS or openAIRE repository & No \\
      \textsf{external\textunderscore source} & Databank:ID of data \textit{e.g.}~\texttt{GEO:GSEXXXXXX} & No \\
      \bottomrule
    \end{tabularx}
    \caption{\textbf{Header Fields.}}\label{tab:header}
  \end{table}
\end{savenotes}
 

\subsection{Data specification}\label{sec:data}

Each~\textbf{data line} contains 11~\textbf{\acs{bedRMod} field}s delimited by a~\textbf{field separator} (tab).
All~\textbf{fields} are mandatory~(\autoref{tab:fields}). Missing data is not allowed.
Additional optional~\textbf{field}s can be added, following the first 11~\textbf{field}s, according to the \acs{BED} specifications, but 
it is not recommended to use \acs{bedRMod} with exactly 12~\textbf{field}s, \textit{cf.}~\autoref{sec:custom_fields}.

\begin{savenotes}
  \begin{table}
    \begin{adjustwidth}{-0.5in}{-0.5in}
      \begin{tabularx}{\linewidth}{r l l l L}
        \toprule
        Col & \acs{bedRMod} Field & Type & Regex or range & Brief description \\
        \midrule
        1
        & \textsf{chrom}
        & String
        & \texttt{[[:alnum:]\_]\{1,255\}}\footnote{\texttt{[[:alnum:]\_]} is equivalent to the \ac{regex}~\texttt{[A-Za-z0-9\_]}. % chktex 8
        It is also equivalent to the Perl extension~\texttt{[[:word:]]}}
        & Chromosome name \\
        2 & \textsf{chromStart} & Int & $[0, 2^{64}-1]$ & \textbf{Feature} start position \\
        3 & \textsf{chromEnd} & Int & $[0, 2^{64} -1]$ & \textbf{Feature} end position \\
        4 
        & \textsf{name} 
        & String 
        & \texttt{[{\textbackslash}x20-{\textbackslash}x7e]\{1,255\}} 
        & Modification name and additional attributes \\
        5 & \textsf{score} & String & \texttt{[{\textbackslash}x20-{\textbackslash}x7e]\{1,255\}} & Modification confidence \\
        6 & \textsf{strand} & String & \texttt{[-+.]} & \textbf{Feature} strand \\
        7 & \textsf{thickStart} & Int & $[0, 2^{64}-1]$ & Thick start position, typically same as~\textsf{chromStart} \\
        8 & \textsf{thickEnd} & Int & $[0, 2^{64}-1]$ & Thick end position, typically same as~\textsf{chromEnd} \\
        9 & \textsf{itemRgb} & Int,Int,Int & \texttt{(}$[0, 255], [0,255], [0,255]$\texttt{) | 0} & Display color \\ % chktex 9
        10 & \textsf{coverage} & Int &  $(0, 2^{64}-1]$ & Coverage \\
        11 & \textsf{frequency} & Float & $[0, 100]$ & Percentage of modification \\
        \bottomrule
      \end{tabularx}
    \end{adjustwidth}
    \caption{\textbf{\acs{bedRMod} Fields.}}\label{tab:fields}
  \end{table}
\end{savenotes}

In a \ac{bedRMod}~\textbf{file}, each~\textbf{data line} must have the same number of~\textbf{field}s.
The positions in~\textbf{\acs{bedRMod} field}s are all described in the~\textbf{0-based, half-open coordinate system}, exactly as 
described in the official \ac{BED} specifications.

\subsection{Coordinates}
Refer to the official \ac{BED} specifications.

\subsection{Simple attributes}
\begin{enumerate}
\item \textsf{name}: String that describes the~\textbf{feature}, \textit{i.e.} the modification. \textsf{name} must describe 
the modification using the \emph{short name} of the MODOMICS nomenclature\footnote{MODOMICS, \url{https://www.genesilico.pl/modomics/modifications}}, or the base modification code described in the SAMtags\footnote{SAM tags, \url{https://samtools.github.io/hts-specs}}, or a numeric ChEBI code\footnote{Chemical Entities of Biological Interest, \url{https://www.ebi.ac.uk/chebi}}. The MODOMICS \emph{short name} corresponding to~\textsf{name} is always described in the~\textbf{header field}~\textsf{modification\textunderscore names}. Additional name attributes are allowed, and must be comma-separated \textit{e.g.}~\texttt{a,DRACH,2}.

\item \textsf{score}: String representation of the confidence in calling this modification. Any measure of confidence is valid, but a \ac{bedRMod}~\textbf{file} with non-integer-like~\textsf{score} values outside the range $[0, 1000]$ may fail from being correctly displayed in a visual representation\footnote{\textit{cf.} bedtools definition of score, \url{https://bedtools.readthedocs.io/en/latest/content/general-usage.html?highlight=bed\%20format}}.

\item \textsf{coverage}: Integer between~0 and the maximum size of an unsigned 64-bit integer, excluding~0, representing the number of reads aligned at this position or the number of calls for this~\textbf{feature}.

\item \textsf{frequency}: Float\footnote{Decimal string representation of 64-bit floating point number, IEEE Standard for Binary Floating-Point Arithmetic. IEEE 754–1985, 1985.} between~0 and~100, including~0, representing the modification frequency, or stoichiometry. This can be the percentage of modified reads, or the ratio of the number of calls passing filters that were classified as a residue with the base modification reported for this~\textbf{feature} to the \emph{valid coverage}, multiplied by~100. See~\autoref{sec:fields} for an explanation of \emph{valid coverage}. A~\textsf{frequency} of~0 means that there is evidence that a given site is not modified, \textit{i.e.} the canonical/unmodified base is reported with a confidence quantified by~\textsf{score}~$>$0.
\end{enumerate}

\subsection{Display attributes}
\begin{enumerate}
  \setcounter{enumi}{4}

\item \textsf{thickStart}: Included for compatibility, typically same as~\textsf{chromStart}.

\item \textsf{thickEnd}: Included for compatibility, typically same as~\textsf{chromEnd}.

\item \textsf{itemRgb}: Included for compatibility, typically~\texttt{0,0,0}.

\end{enumerate}

\subsection{Custom fields}\label{sec:custom_fields}

\textbf{Custom field}s defined by the~\textbf{file} creator may contain any printable 7-bit US \ac{ASCII} character (which includes spaces, but excludes tabs, newlines, and other control characters), as defined by the \ac{BED} format definitions.

A \acs{bedRMod}~\textbf{file} with exactly 12~\textbf{field}s, \textit{i.e.} containing one additional optional~\textbf{field}, may be implicitely 
assumed to be a~\textbf{BED12}~\textbf{file} by certain software and genome browsers, which can result in unexpected behaviour! 

\section{Examples}

\subsection[title]{Example bedRMod file from the \acs{bedRMod} and related specifications\footnote{\url{https://github.com/dieterich-lab/euf-specs/examples/bedrmod/example.bedrmod}}}\label{sec:example-bedrmod}

\begin{verbatim}
#fileformat=bedRModv2
#organism=9606
#modification_type=RNA
#modification_names=20607:m5C:C,21891:m6A:A
#assembly=GRCh38
#annotation_source=Ensembl
#annotation_version=110
#sequencing_platform=Illumina NovaSeq 6000
#basecalling=
#bioinformatics_workflow=workflow:https://github.com/XXX
#experiment=https://doi.org/10.XXX
#external_source=SRA:PRJNAXXXXXX,GEO:GSEXXXXXX
#chrom	chromStart	chromEnd	name	score	strand	thickStart	thickEnd	itemRgb	coverage	frequency
1	1391918	1391919	20607	0	-	1391918	1391919	0,0,0	42	42.56
2	8878712	8878713	20607	0	-	8878712	8878713	0,0,0	318	44.23
3	11980442	11980443	21891	0	+	11980442	11980443	0,0,0	111	56.20
4	17054111	17054112	20607	0	-	17054111	17054112	0,0,0	40	34.03
5	23691799	23691800	21891	0	+	23691799	23691800	0,0,0	352	27.33
\end{verbatim}

\section{Recommended practice for the \acs{bedRMod} format}

\subsection{\acs{bedRMod} extension}
The~\textbf{file} extension is~\texttt{.bedrmod},~\texttt{.bedmethyl}, or~\texttt{.bed}. Since the \ac{BED} format prohibits~\textbf{BED11}, there should be little confusion in general, but when~\textbf{custom field}s are defined, it is recommended to use the~\texttt{.bedrmod} or~\texttt{.bedmethyl} extension. \Ac{bedRMod} formalizes the ENCORE bedMethyl format\footnote{Description of bedMethyl file, \url{https://www.encodeproject.org/data-standards/wgbs}} for \emph{RNA modification}s.

\subsection{Mandatory \acs{bedRMod} header fields}
These~\textbf{header field}s are not free text, and must conform to a controlled vocabulary.
\begin{itemize}
\item \textsf{fileformat}: A valid version of this specification, including the format name, \textit{e.g.}~\texttt{bedRModv2}. 

\item \textsf{modification\textunderscore type}: A valid RNA type\footnote{For example RNA or mRNA, tRNA, or rRNA. A next version of this specification should prescribe a controlled vocabulary by providing a reference to an established RNA ontology.}.

\item \textsf{modification\textunderscore names}: A comma-separated dictionary mapping in the form name:MODOMICS \emph{short name}:reference or unmodified base, where name corresponds to the~\textsf{name}~\textbf{field} of a~\textbf{data line}, MODOMICS \emph{short name} to the corresponding short name of the MODOMICS nomenclature, and reference or unmodified base is the unmodified base as reported by the sequencing instrument for the top strand, as described in the SAMtags, except that `N' is not allowed (since it can be used to match any base). A value is also required when using MODOMICS \emph{short names}, \textit{e.g.}~\texttt{m6A:m6A:A}. All modifications present in the~\textbf{file} must be included, in a comma-separated list of items.

\item \textsf{organism}: A valid NCBI Taxonomic identifier, \textit{e.g.}~\texttt{9606}.

\item \textsf{assembly}: The name of a valid assembly, \textit{e.g.} using the Ensembl terminology,~\texttt{GRCh38}. 

\item \textsf{annotation\textunderscore source}: The name of a valid annotation, \textit{e.g.}~\texttt{Ensembl}. 

\item \textsf{annotation\textunderscore version}: A valid version for the annotation source, \textit{e.g.}~\texttt{110}. 
\end{itemize}

\subsection{\acs{bedRMod} header fields}
These~\textbf{header field}s can be left without a value, but the key must always be present. The value is free text, although it is strongly recommended
to reference established \textsf{sequencing\textunderscore platform}s, \textsf{basecalling} models, \textsf{bioinformatics\textunderscore workflow}s, or \textsf{external\textunderscore source}s using an exact terminology and/or recognized identifiers.  
\begin{itemize}
\item \textsf{sequencing\textunderscore platform}: Typically, the name of the sequencing instrument or device, including key specifications if relevant, \textit{e.g.}~\texttt{ONT MinION}.

\item \textsf{basecalling}: Basecalling model such as name of versioned model, reference to published model, and/or additional details on training, \textit{e.g.}~\texttt{dna\textunderscore r9.4.1\textunderscore e8\textunderscore sup$@$v3.6}.

\item \textsf{bioinformatics\textunderscore workflow}: Program name, version, and/or call used to generate the~\textbf{file}, or link to open source bioinformatics workflow, including version and/or any additional details to facilitate data lineage. The information should be sufficient to reproduce the content of the~\textbf{file}.

\item \textsf{experiment}: Supplementary information about experimental protocol, design, or the content of the~\textbf{file} such as conditions used, number of replicates, \textit{etc.}, or link to an openAIRE repository.

\item \textsf{external\textunderscore source}: A comma-separated list of sources of the form Databank:ID, \textit{e.g.}~~\texttt{GEO:GSEXXXXXX, Zenodo:10.XXX/zenodo.XXXXXXXX}. Free text is allowed, but this specification recommends using the format described here. Together with~\textsf{bioinformatics\textunderscore workflow}, this should allow to reproduce the content of the~\textbf{file}.
\end{itemize}


\subsection{\acs{bedRMod} fields}\label{sec:fields}
\begin{itemize}
\item \textsf{chrom}: The name of each chromosome should match the names from a reference genome assembly, as given in the~\textbf{header}.
  For example, if~\texttt{\#assembly=GRCh38}, then chromosomes should be named~\texttt{1} to~\texttt{22}, \texttt{X}, \texttt{Y}, and~\texttt{MT},
  consistently through the~\textbf{file}.
\item \textsf{name}: The MODOMICS \emph{short names} or the ChEBI codes should be used in preference to the base modification codes described in the SAMtags.
\item \textsf{score}: The \emph{valid coverage} should be used as a measure of confidence. This can be \textit{e.g.} the number of calls passing filters (classified as modified and unmodified) at the reported modification position\footnote{\textit{cf.} \url{https://nanoporetech.github.io/modkit}}, or the number or reads remaining after filtering and used to infer the modificaton status. If \emph{valid coverage} is not available, then~\textsf{coverage} can be used as a measure of confidence.
\end{itemize}

\subsection{Whitespace}\label{sec:whitespace}
We recommend that only a single tab~(\texttt{{\textbackslash}t}) be used as~\textbf{field separator}, \textit{cf.} offical \ac{BED} specifications.

\section{Information supplied out-of-band}

A \ac{bedRMod} \textbf{file} contains 11 required~\textbf{field}s, any additional~\textbf{field}s may require information that must be supplied out-of-band.
A common practice is to include a~\textbf{comment line} after the~\textbf{header} to describe the~\textbf{field}s used in the~\textbf{file}, \textit{cf.}~\autoref{sec:example-bedrmod}.

The semantics of~\textbf{field}s such as~\textsf{score},~\textsf{coverage}, and~\textsf{frequency} can be included in the~\textsf{header} using the \textbf{header field}~\texttt{bioinformatics\textunderscore workflow}.

\section{Acronyms}

% using the optional argument to acronym to set the label width causes it to use the list environment instead of description, which means we can't set nosep easily
\setlist[description]{labelwidth=\widthof{\textbf{\acs{bedRMod}}},nosep}
\begin{acronym}
  \acro{ASCII}{American Standard Code for Information Interchange}
  \acro{BED}{Browser Extensible Data}
  \acro{bedRMod}{Browser Extensible Data for RNA modification}
  \acro{ChEBI}{Chemical Entities of Biological Interest}
  \acro{NCBI}{National Center for Biotechnology Information}
  \acro{regex}{regular expression}
\end{acronym}

\section{Acknowledgments}

We thank the \acf{bedRMod} format specification working group and the Modkit developers.

\end{document}

% chktex-file 17

%%% Local Variables:
%%% mode: latex
%%% TeX-master: t
%%% End:


\hypersetup{colorlinks=true,
  linkcolor=blue,
  filecolor=magenta,
  urlcolor=blue,
  pdfinfo={githash=\commitdesc}}

\definecolor{cverbbg}{gray}{0.93}

\title{The \acf{bedRMod} format}
\author{Transregio 319 RMaP}
\date{\headdate}

\setlength{\emergencystretch}{\hsize}
\setlength{\footnotemargin}{1em}

\floatplacement{table}{htbp}
\setcounter{topnumber}{2}
\setcounter{bottomnumber}{2}
\setcounter{totalnumber}{4}
\setcounter{dbltopnumber}{2}
\renewcommand{\dbltopfraction}{0.9}
\renewcommand{\textfraction}{0.07}
\renewcommand{\floatpagefraction}{0.7}

\interfootnotelinepenalty=1000000
\makesavenoteenv{tabularx}

\newcolumntype{L}{>{\raggedright\arraybackslash}X}

\providecommand*{\Ac}[1]{\ac{#1}} % work around outdated acronym.sty packages
\newcommand*{\acrodefused}[2]{\acrodef{#1}{#2}\acused{#1}}

\frenchspacing

% eliminate passive voice warnings
% chktex-file -3

\begin{document}

\maketitle

\begin{small}
\noindent
The master version of this document can be found at \url{https://github.com/dieterich-lab/euf-specs}.
This printing is version~\commitdesc\ from that repository, last modified on the date shown above.
\end{small}

\acused{ASCII}

\section{Specification}

\Ac{bedRMod} is a tab-delimited~\textbf{file} format, compatible with the \acf{BED} format\footnote{SAM/BAM and related specifications, \url{http://samtools.github.io/hts-specs}}. Metadata are in~\textbf{header line}s, which describe metainformation about the source of the data. Data are in~\textbf{data line}s, which describe \emph{RNA modification}s, or sites of putative \emph{RNA modification}s, by physical start and end position on a linear chromosome. The metadata must be consistent for all~\textbf{data line}s. The~\textbf{file} extension for the \ac{bedRMod} format is~\texttt{.bedrmod},~\texttt{.bedmethyl}, or~\texttt{.bed}.

\subsection{Scope}

This specification is a variation of the \ac{BED} description for~\textbf{data line}s. The content of this document is directly inspired from the 
official \ac{BED} specifications. Only the most important or less obvious concepts are reiterated in this document. For general information, refer to the official \ac{BED} specifications. Potential interoperability issues with the \ac{BED} format are also described in this document.

\subsection{Typographic conventions}

This document uses the official \ac{BED} typographic conventions~(\autoref{tab:typographic-conventions}).

\begin{savenotes}
  \begin{table}
    \begin{tabularx}{\textwidth}{r L L}
      \toprule
      Style & Meaning & Examples \\
      \midrule
      Bold & Terms defined in subsections~\ref{sec:terms}--\ref{sec:lines} & \textbf{file}{\quad}\textbf{line} \\
      Sans serif & Names of~\textbf{field}s & \textsf{chrom}{\quad}\textsf{chromStart}{\quad}\textsf{chromEnd} \\
      Fixed-width & Literals or \ac{regex}es\footnote{POSIX/IEEE~1003.1--2017 Extended Regular Expressions, for the ``C'' locale.
                    \emph{IEEE Standard for Information Technology---Portable Operating System Interface~(POSIX) Base Specifications}, IEEE~1003.1--2017, 2017} & \texttt{.bedrmod}{\quad}\texttt{grep}{\quad}\texttt{[[:alnum:]]+} \\
      \bottomrule
    \end{tabularx}
    \caption{\textbf{Typographic conventions.}}\label{tab:typographic-conventions}
  \end{table}
\end{savenotes}

\subsection{Terminology and concepts}\label{sec:terms}
\begin{description}
\item[0-based, half-open coordinate system:]
  A coordinate system where the first base starts at position~0, and the start of the interval is included but the end is not.
  For example, for a sequence of bases~\texttt{ACTGCG}, the bases given by the interval~[2,~4) are~\texttt{TG}. % chktex 9

\item[\acs{bedRMod} field:]
  One of the 11~standard~\textbf{field}s defined in this specification.
  All~\textbf{\acs{bedRMod} field}s are mandatory.

\item[comment line:]
  A~\textbf{line} that starts with~\texttt{\#} with no horizontal whitespace beforehand. \textbf{Comment line}s at the start of 
  the~\textbf{file} are~\textbf{header line}s defined in this specification.

\item[custom field:]
  A~\textbf{field} defined by the~\textbf{file}~creator.
  \textbf{Custom field}s occur in each~\textbf{line} after any~\textbf{\acs{bedRMod} field}s.

\item[data line:]
  A~\textbf{line} that contains~\textbf{feature}~data.

\item[feature:]
  A linear region of a chromosome reporting the presence or absence of a given RNA modification, supported by quantitative evidence, typically at single-base resolution, but can include a context.

\item[field:]
  Data stored as non-tab text.
  All~\textbf{field}s are 7-bit US \ac{ASCII} printable characters\footnote{Characters in the range~\texttt{{\textbackslash}x20} to~\texttt{{\textbackslash}x7e}, therefore not including any control characters}.

\item[field separator:]
  One or more horizontal whitespace characters (space or tab).
  The~\textbf{field separator} must match the \ac{regex}~\texttt{[ {\textbackslash}t]+}.
  This specification strongly recommends using tab as~\textbf{field separator} throughout the~\textbf{file}.

\item[file:]
  Sequence of one or more~\textbf{data line}s with a~\textbf{header}.

\item[header:]
  Mandatory~\textbf{header line}s, followed by optional~\textbf{comment line}s, at the start of the~\textbf{file}.

\item[header field:]
  A mandatory~\emph{key=value} pair describing one of the~\textbf{header line}s. 
  
\item[header line:]
  A~\textbf{line} that starts with~\texttt{\#} with no horizontal whitespace beforehand, immediately followed by a~\textbf{header field}.
  
\item[line:]
  String terminated by a~\textbf{line separator}, in one of the following classes.
  Either a~\textbf{data line} or a~\textbf{comment line}, \textit{cf.}~\autoref{sec:lines}.

\item[line separator:]
  Either carriage return~(\texttt{{\textbackslash}r}, equivalent to~\texttt{{\textbackslash}x0d}), newline~(\texttt{{\textbackslash}n}, equivalent to~\texttt{{\textbackslash}x0a}), or carriage return followed by newline~(\texttt{{\textbackslash}r{\textbackslash}n}, equivalent to~\texttt{{\textbackslash}x0d{\textbackslash}x0a}).
  The same~\textbf{line separator} must be used throughout the~\textbf{file}.
\end{description}

\subsection{Lines}\label{sec:lines}

\subsubsection{Data lines}

\textbf{Data line}s contain~\textbf{feature}~data (RNA modification).
A~\textbf{data line} is composed of~\textbf{field}s separated by~\textbf{field separator}s.

\subsubsection{Comment lines}

\textbf{Comment line}s provide no~\textbf{feature} data. They start with~\texttt{\#} with no horizontal whitespace beforehand.
\textbf{Comment line}s at the beginning of the~\textbf{file} are treated as~\textbf{header line}s, and must conform to~\textbf{header} specifications, \textit{cf.}~\autoref{sec:header}. A~\texttt{\#} appearing anywhere else in a~\textbf{data line} is treated as~\textbf{feature} data, not a comment.


\subsection{Header specification}\label{sec:header}

The~\textbf{header} contains metainformation about the source of the data. Each~\textbf{header line} starts with a~\texttt{\#} and contains a
mandatory~\textbf{header field} in the form of a~\emph{key=value} pair, \textit{e.g.}~\texttt{\#fileformat=bedRModv2}~(\autoref{tab:header}). All~\textbf{header field}s are mandatory. The first seven~\textbf{header field}s must be assigned a value, and the value must generally follow a controlled vocabulary; the remaining~\textbf{header field}s are free text, and can be left without a value, although it is strongly advised to provide a value for each one. Additional~\textbf{line}s starting with~\texttt{\#} are treated as~\textbf{comment line}s.

A \ac{bedRMod}~\textbf{header} describes information for one organism, one assembly and annotation, and one modification (RNA) type, hence a \ac{bedRMod}~\textbf{file} contains~\textbf{data lines} for one organism, one assembly and annotation, and one modification (RNA) type. A \ac{bedRMod}~\textbf{file} can contain~\textbf{data lines} for different RNA modifications, \textit{e.g.}~\texttt{m6A} and~\texttt{m5C}, \textit{cf.}~\autoref{sec:data}.


\begin{savenotes}
  \begin{table}
    \begin{tabularx}{\textwidth}{X p{.5\textwidth} p{.15\textwidth}}
      \toprule
      Header Field & Brief description & Value required \\
      \midrule
      \textsf{fileformat} & Fileformat and version \textit{e.g.}~\texttt{bedRModv2} & Yes \\
      \textsf{organism} & NCBI Taxonomic identifier\footnote{NCBI Taxonomy: a comprehensive update on curation, resources and tools, \url{10.1093/database/baaa062}} & Yes \\
      \textsf{modification\textunderscore type} & RNA & Yes \\
      \textsf{modification\textunderscore names} & name:MODOMICS \emph{short name}:reference or unmodified base \textit{e.g.}~\texttt{21891:m6A:A} & Yes \\
      \textsf{assembly} & Genome or transcriptome assembly \textit{e.g.}~\texttt{GRCh38} & Yes \\
      \textsf{annotation\textunderscore source} & Annotation source \textit{e.g.}~\texttt{Ensembl} & Yes \\
      \textsf{annotation\textunderscore version} & Annotation version \textit{e.g.}~\texttt{110} & Yes \\
      \textsf{sequencing\textunderscore platform} & Sequencing platform \textit{e.g.}~\texttt{Illumina NovaSeq 6000}, or~\texttt{ONT MinION} & No \\
      \textsf{basecalling} & Basecalling model information where relevant & No \\
      \textsf{bioinformatics\textunderscore workflow} & Link to bioinformatics workflow, program name, version, and/or call, or information relevant to score, coverage, or frequency calculation, \textit{etc.} & No \\
      \textsf{experiment} & Information about experimental protocol and design, or link to \textit{e.g.} openBIS or openAIRE repository & No \\
      \textsf{external\textunderscore source} & Databank:ID of data \textit{e.g.}~\texttt{GEO:GSEXXXXXX} & No \\
      \bottomrule
    \end{tabularx}
    \caption{\textbf{Header Fields.}}\label{tab:header}
  \end{table}
\end{savenotes}
 

\subsection{Data specification}\label{sec:data}

Each~\textbf{data line} contains 11~\textbf{\acs{bedRMod} field}s delimited by a~\textbf{field separator} (tab).
All~\textbf{fields} are mandatory~(\autoref{tab:fields}). Missing data is not allowed.
Additional optional~\textbf{field}s can be added, following the first 11~\textbf{field}s, according to the \acs{BED} specifications, but 
it is not recommended to use \acs{bedRMod} with exactly 12~\textbf{field}s, \textit{cf.}~\autoref{sec:custom_fields}.

\begin{savenotes}
  \begin{table}
    \begin{adjustwidth}{-0.5in}{-0.5in}
      \begin{tabularx}{\linewidth}{r l l l L}
        \toprule
        Col & \acs{bedRMod} Field & Type & Regex or range & Brief description \\
        \midrule
        1
        & \textsf{chrom}
        & String
        & \texttt{[[:alnum:]\_]\{1,255\}}\footnote{\texttt{[[:alnum:]\_]} is equivalent to the \ac{regex}~\texttt{[A-Za-z0-9\_]}. % chktex 8
        It is also equivalent to the Perl extension~\texttt{[[:word:]]}}
        & Chromosome name \\
        2 & \textsf{chromStart} & Int & $[0, 2^{64}-1]$ & \textbf{Feature} start position \\
        3 & \textsf{chromEnd} & Int & $[0, 2^{64} -1]$ & \textbf{Feature} end position \\
        4 
        & \textsf{name} 
        & String 
        & \texttt{[{\textbackslash}x20-{\textbackslash}x7e]\{1,255\}} 
        & Modification name and additional attributes \\
        5 & \textsf{score} & String & \texttt{[{\textbackslash}x20-{\textbackslash}x7e]\{1,255\}} & Modification confidence \\
        6 & \textsf{strand} & String & \texttt{[-+.]} & \textbf{Feature} strand \\
        7 & \textsf{thickStart} & Int & $[0, 2^{64}-1]$ & Thick start position, typically same as~\textsf{chromStart} \\
        8 & \textsf{thickEnd} & Int & $[0, 2^{64}-1]$ & Thick end position, typically same as~\textsf{chromEnd} \\
        9 & \textsf{itemRgb} & Int,Int,Int & \texttt{(}$[0, 255], [0,255], [0,255]$\texttt{) | 0} & Display color \\ % chktex 9
        10 & \textsf{coverage} & Int &  $(0, 2^{64}-1]$ & Coverage \\
        11 & \textsf{frequency} & Float & $[0, 100]$ & Percentage of modification \\
        \bottomrule
      \end{tabularx}
    \end{adjustwidth}
    \caption{\textbf{\acs{bedRMod} Fields.}}\label{tab:fields}
  \end{table}
\end{savenotes}

In a \ac{bedRMod}~\textbf{file}, each~\textbf{data line} must have the same number of~\textbf{field}s.
The positions in~\textbf{\acs{bedRMod} field}s are all described in the~\textbf{0-based, half-open coordinate system}, exactly as 
described in the official \ac{BED} specifications.

\subsection{Coordinates}
Refer to the official \ac{BED} specifications.

\subsection{Simple attributes}
\begin{enumerate}
\item \textsf{name}: String that describes the~\textbf{feature}, \textit{i.e.} the modification. \textsf{name} must describe 
the modification using the \emph{short name} of the MODOMICS nomenclature\footnote{MODOMICS, \url{https://www.genesilico.pl/modomics/modifications}}, or the base modification code described in the SAMtags\footnote{SAM tags, \url{https://samtools.github.io/hts-specs}}, or a numeric ChEBI code\footnote{Chemical Entities of Biological Interest, \url{https://www.ebi.ac.uk/chebi}}. The MODOMICS \emph{short name} corresponding to~\textsf{name} is always described in the~\textbf{header field}~\textsf{modification\textunderscore names}. Additional name attributes are allowed, and must be comma-separated \textit{e.g.}~\texttt{a,DRACH,2}.

\item \textsf{score}: String representation of the confidence in calling this modification. Any measure of confidence is valid, but a \ac{bedRMod}~\textbf{file} with non-integer-like~\textsf{score} values outside the range $[0, 1000]$ may fail from being correctly displayed in a visual representation\footnote{\textit{cf.} bedtools definition of score, \url{https://bedtools.readthedocs.io/en/latest/content/general-usage.html?highlight=bed\%20format}}.

\item \textsf{coverage}: Integer between~0 and the maximum size of an unsigned 64-bit integer, excluding~0, representing the number of reads aligned at this position or the number of calls for this~\textbf{feature}.

\item \textsf{frequency}: Float\footnote{Decimal string representation of 64-bit floating point number, IEEE Standard for Binary Floating-Point Arithmetic. IEEE 754–1985, 1985.} between~0 and~100, including~0, representing the modification frequency, or stoichiometry. This can be the percentage of modified reads, or the ratio of the number of calls passing filters that were classified as a residue with the base modification reported for this~\textbf{feature} to the \emph{valid coverage}, multiplied by~100. See~\autoref{sec:fields} for an explanation of \emph{valid coverage}. A~\textsf{frequency} of~0 means that there is evidence that a given site is not modified, \textit{i.e.} the canonical/unmodified base is reported with a confidence quantified by~\textsf{score}~$>$0.
\end{enumerate}

\subsection{Display attributes}
\begin{enumerate}
  \setcounter{enumi}{4}

\item \textsf{thickStart}: Included for compatibility, typically same as~\textsf{chromStart}.

\item \textsf{thickEnd}: Included for compatibility, typically same as~\textsf{chromEnd}.

\item \textsf{itemRgb}: Included for compatibility, typically~\texttt{0,0,0}.

\end{enumerate}

\subsection{Custom fields}\label{sec:custom_fields}

\textbf{Custom field}s defined by the~\textbf{file} creator may contain any printable 7-bit US \ac{ASCII} character (which includes spaces, but excludes tabs, newlines, and other control characters), as defined by the \ac{BED} format definitions.

A \acs{bedRMod}~\textbf{file} with exactly 12~\textbf{field}s, \textit{i.e.} containing one additional optional~\textbf{field}, may be implicitely 
assumed to be a~\textbf{BED12}~\textbf{file} by certain software and genome browsers, which can result in unexpected behaviour! 

\section{Examples}

\subsection[title]{Example bedRMod file from the \acs{bedRMod} and related specifications\footnote{\url{https://github.com/dieterich-lab/euf-specs/examples/bedrmod/example.bedrmod}}}\label{sec:example-bedrmod}

\begin{verbatim}
#fileformat=bedRModv2
#organism=9606
#modification_type=RNA
#modification_names=20607:m5C:C,21891:m6A:A
#assembly=GRCh38
#annotation_source=Ensembl
#annotation_version=110
#sequencing_platform=Illumina NovaSeq 6000
#basecalling=
#bioinformatics_workflow=workflow:https://github.com/XXX
#experiment=https://doi.org/10.XXX
#external_source=SRA:PRJNAXXXXXX,GEO:GSEXXXXXX
#chrom	chromStart	chromEnd	name	score	strand	thickStart	thickEnd	itemRgb	coverage	frequency
1	1391918	1391919	20607	0	-	1391918	1391919	0,0,0	42	42.56
2	8878712	8878713	20607	0	-	8878712	8878713	0,0,0	318	44.23
3	11980442	11980443	21891	0	+	11980442	11980443	0,0,0	111	56.20
4	17054111	17054112	20607	0	-	17054111	17054112	0,0,0	40	34.03
5	23691799	23691800	21891	0	+	23691799	23691800	0,0,0	352	27.33
\end{verbatim}

\section{Recommended practice for the \acs{bedRMod} format}

\subsection{\acs{bedRMod} extension}
The~\textbf{file} extension is~\texttt{.bedrmod},~\texttt{.bedmethyl}, or~\texttt{.bed}. Since the \ac{BED} format prohibits~\textbf{BED11}, there should be little confusion in general, but when~\textbf{custom field}s are defined, it is recommended to use the~\texttt{.bedrmod} or~\texttt{.bedmethyl} extension. \Ac{bedRMod} formalizes the ENCORE bedMethyl format\footnote{Description of bedMethyl file, \url{https://www.encodeproject.org/data-standards/wgbs}} for \emph{RNA modification}s.

\subsection{Mandatory \acs{bedRMod} header fields}
These~\textbf{header field}s are not free text, and must conform to a controlled vocabulary.
\begin{itemize}
\item \textsf{fileformat}: A valid version of this specification, including the format name, \textit{e.g.}~\texttt{bedRModv2}. 

\item \textsf{modification\textunderscore type}: A valid RNA type\footnote{For example RNA or mRNA, tRNA, or rRNA. A next version of this specification should prescribe a controlled vocabulary by providing a reference to an established RNA ontology.}.

\item \textsf{modification\textunderscore names}: A comma-separated dictionary mapping in the form name:MODOMICS \emph{short name}:reference or unmodified base, where name corresponds to the~\textsf{name}~\textbf{field} of a~\textbf{data line}, MODOMICS \emph{short name} to the corresponding short name of the MODOMICS nomenclature, and reference or unmodified base is the unmodified base as reported by the sequencing instrument for the top strand, as described in the SAMtags, except that `N' is not allowed (since it can be used to match any base). A value is also required when using MODOMICS \emph{short names}, \textit{e.g.}~\texttt{m6A:m6A:A}. All modifications present in the~\textbf{file} must be included, in a comma-separated list of items.

\item \textsf{organism}: A valid NCBI Taxonomic identifier, \textit{e.g.}~\texttt{9606}.

\item \textsf{assembly}: The name of a valid assembly, \textit{e.g.} using the Ensembl terminology,~\texttt{GRCh38}. 

\item \textsf{annotation\textunderscore source}: The name of a valid annotation, \textit{e.g.}~\texttt{Ensembl}. 

\item \textsf{annotation\textunderscore version}: A valid version for the annotation source, \textit{e.g.}~\texttt{110}. 
\end{itemize}

\subsection{\acs{bedRMod} header fields}
These~\textbf{header field}s can be left without a value, but the key must always be present. The value is free text, although it is strongly recommended
to reference established \textsf{sequencing\textunderscore platform}s, \textsf{basecalling} models, \textsf{bioinformatics\textunderscore workflow}s, or \textsf{external\textunderscore source}s using an exact terminology and/or recognized identifiers.  
\begin{itemize}
\item \textsf{sequencing\textunderscore platform}: Typically, the name of the sequencing instrument or device, including key specifications if relevant, \textit{e.g.}~\texttt{ONT MinION}.

\item \textsf{basecalling}: Basecalling model such as name of versioned model, reference to published model, and/or additional details on training, \textit{e.g.}~\texttt{dna\textunderscore r9.4.1\textunderscore e8\textunderscore sup$@$v3.6}.

\item \textsf{bioinformatics\textunderscore workflow}: Program name, version, and/or call used to generate the~\textbf{file}, or link to open source bioinformatics workflow, including version and/or any additional details to facilitate data lineage. The information should be sufficient to reproduce the content of the~\textbf{file}.

\item \textsf{experiment}: Supplementary information about experimental protocol, design, or the content of the~\textbf{file} such as conditions used, number of replicates, \textit{etc.}, or link to an openAIRE repository.

\item \textsf{external\textunderscore source}: A comma-separated list of sources of the form Databank:ID, \textit{e.g.}~~\texttt{GEO:GSEXXXXXX, Zenodo:10.XXX/zenodo.XXXXXXXX}. Free text is allowed, but this specification recommends using the format described here. Together with~\textsf{bioinformatics\textunderscore workflow}, this should allow to reproduce the content of the~\textbf{file}.
\end{itemize}


\subsection{\acs{bedRMod} fields}\label{sec:fields}
\begin{itemize}
\item \textsf{chrom}: The name of each chromosome should match the names from a reference genome assembly, as given in the~\textbf{header}.
  For example, if~\texttt{\#assembly=GRCh38}, then chromosomes should be named~\texttt{1} to~\texttt{22}, \texttt{X}, \texttt{Y}, and~\texttt{MT},
  consistently through the~\textbf{file}.
\item \textsf{name}: The MODOMICS \emph{short names} or the ChEBI codes should be used in preference to the base modification codes described in the SAMtags.
\item \textsf{score}: The \emph{valid coverage} should be used as a measure of confidence. This can be \textit{e.g.} the number of calls passing filters (classified as modified and unmodified) at the reported modification position\footnote{\textit{cf.} \url{https://nanoporetech.github.io/modkit}}, or the number or reads remaining after filtering and used to infer the modificaton status. If \emph{valid coverage} is not available, then~\textsf{coverage} can be used as a measure of confidence.
\end{itemize}

\subsection{Whitespace}\label{sec:whitespace}
We recommend that only a single tab~(\texttt{{\textbackslash}t}) be used as~\textbf{field separator}, \textit{cf.} offical \ac{BED} specifications.

\section{Information supplied out-of-band}

A \ac{bedRMod} \textbf{file} contains 11 required~\textbf{field}s, any additional~\textbf{field}s may require information that must be supplied out-of-band.
A common practice is to include a~\textbf{comment line} after the~\textbf{header} to describe the~\textbf{field}s used in the~\textbf{file}, \textit{cf.}~\autoref{sec:example-bedrmod}.

The semantics of~\textbf{field}s such as~\textsf{score},~\textsf{coverage}, and~\textsf{frequency} can be included in the~\textsf{header} using the \textbf{header field}~\texttt{bioinformatics\textunderscore workflow}.

\section{Acronyms}

% using the optional argument to acronym to set the label width causes it to use the list environment instead of description, which means we can't set nosep easily
\setlist[description]{labelwidth=\widthof{\textbf{\acs{bedRMod}}},nosep}
\begin{acronym}
  \acro{ASCII}{American Standard Code for Information Interchange}
  \acro{BED}{Browser Extensible Data}
  \acro{bedRMod}{Browser Extensible Data for RNA modification}
  \acro{ChEBI}{Chemical Entities of Biological Interest}
  \acro{NCBI}{National Center for Biotechnology Information}
  \acro{regex}{regular expression}
\end{acronym}

\section{Acknowledgments}

We thank the \acf{bedRMod} format specification working group and the Modkit developers.

\end{document}

% chktex-file 17

%%% Local Variables:
%%% mode: latex
%%% TeX-master: t
%%% End:


\hypersetup{colorlinks=true,
  linkcolor=blue,
  filecolor=magenta,
  urlcolor=blue,
  pdfinfo={githash=\commitdesc}}

\definecolor{cverbbg}{gray}{0.93}

\title{The \acf{bedRMod} format}
\author{Transregio 319 RMaP}
\date{\headdate}

\setlength{\emergencystretch}{\hsize}
\setlength{\footnotemargin}{1em}

\floatplacement{table}{htbp}
\setcounter{topnumber}{2}
\setcounter{bottomnumber}{2}
\setcounter{totalnumber}{4}
\setcounter{dbltopnumber}{2}
\renewcommand{\dbltopfraction}{0.9}
\renewcommand{\textfraction}{0.07}
\renewcommand{\floatpagefraction}{0.7}

\interfootnotelinepenalty=1000000
\makesavenoteenv{tabularx}

\newcolumntype{L}{>{\raggedright\arraybackslash}X}

\providecommand*{\Ac}[1]{\ac{#1}} % work around outdated acronym.sty packages
\newcommand*{\acrodefused}[2]{\acrodef{#1}{#2}\acused{#1}}

\frenchspacing

% eliminate passive voice warnings
% chktex-file -3

\begin{document}

\maketitle

\begin{small}
\noindent
The master version of this document can be found at \url{https://github.com/dieterich-lab/euf-specs}.
This printing is version~\commitdesc\ from that repository, last modified on the date shown above.
\end{small}

\acused{ASCII}

\section{Specification}

\Ac{bedRMod} formalizes the ENCODE bedMethyl format\footnote{Description of bedMethyl file, \url{https://www.encodeproject.org/data-standards/wgbs}} for \emph{RNA modification}s. \Ac{bedRMod} is a tab-delimited~\textbf{file} format, compatible with the \acf{BED} format\footnote{SAM/BAM and related specifications, \url{http://samtools.github.io/hts-specs}}. Metadata are in~\textbf{header line}s, which describe metainformation about the source of the data. Data are in~\textbf{data line}s, which describe \emph{RNA modification}s, or sites of putative \emph{RNA modification}s, by physical start and end position on a linear chromosome. The metadata must be consistent for all~\textbf{data line}s. The~\textbf{file} extension for the \ac{bedRMod} format is~\texttt{.bedrmod},~\texttt{.bedmethyl}, or~\texttt{.bed}.

\subsection{Scope}

This specification is a variation of the \ac{BED} description for~\textbf{data line}s. The content of this document is directly inspired from the 
official \ac{BED} specifications. Only the most important or less obvious concepts are reiterated in this document. For general information, refer to the official \ac{BED} specifications. Potential interoperability issues with the \ac{BED} format are also described in this document.

\subsection{Typographic conventions}

This document uses the official \ac{BED} typographic conventions~(\autoref{tab:typographic-conventions}).

\begin{savenotes}
  \begin{table}
    \begin{tabularx}{\textwidth}{r L L}
      \toprule
      Style & Meaning & Examples \\
      \midrule
      Bold & Terms defined in~\autoref{sec:terms} and~\ref{sec:lines} & \textbf{file}{\quad}\textbf{line} \\
      Sans serif & Names of~\textbf{field}s & \textsf{chrom}{\quad}\textsf{chromStart}{\quad}\textsf{chromEnd} \\
      Fixed-width & Literals or \ac{regex}es\footnote{POSIX/IEEE~1003.1--2017 Extended Regular Expressions, for the ``C'' locale.
                    \emph{IEEE Standard for Information Technology---Portable Operating System Interface~(POSIX) Base Specifications}, IEEE~1003.1--2017, 2017} & \texttt{.bedrmod}{\quad}\texttt{grep}{\quad}\texttt{[[:alnum:]]+} \\
      \bottomrule
    \end{tabularx}
    \caption{\textbf{Typographic conventions.}}\label{tab:typographic-conventions}
  \end{table}
\end{savenotes}

\subsection{Terminology and concepts}\label{sec:terms}
\begin{description}
\item[0-based, half-open coordinate system:]
  A coordinate system where the first base starts at position~0, and the start of the interval is included but the end is not.
  For example, for a sequence of bases~\texttt{ACTGCG}, the bases given by the interval~[2,~4) are~\texttt{TG}. % chktex 9

\item[\acs{bedRMod} field:]
  One of the 11~standard~\textbf{field}s defined in this specification.
  All~\textbf{\acs{bedRMod} field}s are mandatory.

\item[comment line:]
  A~\textbf{line} that starts with~\texttt{\#} with no horizontal whitespace beforehand. \textbf{Comment line}s at the start of 
  the~\textbf{file} are~\textbf{header line}s defined in this specification.

\item[custom field:]
  A~\textbf{field} defined by the~\textbf{file}~creator.
  \textbf{Custom field}s occur in each~\textbf{line} after any~\textbf{\acs{bedRMod} field}s.

\item[data line:]
  A~\textbf{line} that contains~\textbf{feature}~data.

\item[feature:]
  A linear region of a chromosome reporting a \emph{RNA modification}, or a site of putative \emph{RNA modification}, supported by quantitative evidence, typically at single-base resolution, but can include a context.

\item[field:]
  Data stored as non-tab text.
  All~\textbf{field}s are 7-bit US \ac{ASCII} printable characters\footnote{Characters in the range~\texttt{{\textbackslash}x20} to~\texttt{{\textbackslash}x7e}, therefore not including any control characters}.

\item[field separator:]
  One or more horizontal whitespace characters (space or tab).
  The~\textbf{field separator} must match the \ac{regex}~\texttt{[ {\textbackslash}t]+}.
  This specification strongly recommends using tab as~\textbf{field separator} throughout the~\textbf{file}.

\item[file:]
  Sequence of one or more~\textbf{data line}s with a~\textbf{header}.

\item[header:]
  Mandatory~\textbf{header line}s, followed by optional~\textbf{comment line}s, at the start of the~\textbf{file}, \textit{cf.}~\autoref{sec:header}.

\item[header field:]
  A mandatory~\emph{key=value} pair describing one of the~\textbf{header line}s. 
  
\item[header line:]
  A~\textbf{line} that starts with~\texttt{\#} with no horizontal whitespace beforehand, immediately followed by a~\textbf{header field}.
  
\item[line:]
  String terminated by a~\textbf{line separator}, in one of the following classes.
  Either a~\textbf{data line} or a~\textbf{comment line}, \textit{cf.}~\autoref{sec:lines}.

\item[line separator:]
  Either carriage return~(\texttt{{\textbackslash}r}, equivalent to~\texttt{{\textbackslash}x0d}), newline~(\texttt{{\textbackslash}n}, equivalent to~\texttt{{\textbackslash}x0a}), or carriage return followed by newline~(\texttt{{\textbackslash}r{\textbackslash}n}, equivalent to~\texttt{{\textbackslash}x0d{\textbackslash}x0a}).
  The same~\textbf{line separator} must be used throughout the~\textbf{file}.
\end{description}

\subsection{Lines}\label{sec:lines}

\subsubsection{Data lines}

\textbf{Data line}s contain~\textbf{feature}~data.
A~\textbf{data line} is composed of~\textbf{field}s separated by~\textbf{field separator}s.

\subsubsection{Comment lines}

\textbf{Comment line}s provide no~\textbf{feature} data. They start with~\texttt{\#} with no horizontal whitespace beforehand.
\textbf{Comment line}s at the beginning of the~\textbf{file} are treated as~\textbf{header line}s, and must conform to~\textbf{header} specifications. A~\texttt{\#} appearing anywhere else in a~\textbf{data line} is treated as~\textbf{feature} data, not a comment.


\subsection{Header specification}\label{sec:header}

The~\textbf{header} contains metainformation about the source of the data. Each~\textbf{header line} starts with a~\texttt{\#} and contains a
mandatory~\textbf{header field} in the form of a~\emph{key=value} pair~(\autoref{tab:header}). All~\textbf{header field}s are mandatory. The first seven~\textbf{header field}s must be assigned a value, and the value must follow a controlled vocabulary, see~\autoref{sec:recommend-header} and~\ref{sec:recommend-fields} for examples and recommendations. Additional~\textbf{line}s starting with~\texttt{\#} are treated as~\textbf{comment line}s.

A \ac{bedRMod}~\textbf{header} describes information for one organism, one assembly and annotation, and one modification (RNA) type, hence a \ac{bedRMod}~\textbf{file} contains~\textbf{data lines} for one organism, one assembly and annotation, and one modification (RNA) type, \textit{e.g.} a \ac{bedRMod}~\textbf{file} contains~\textbf{data lines} for~\texttt{m6A} and~\texttt{m5C} in human mRNA, using~\texttt{GRCh38} and ~\texttt{Ensembl 110}.

\begin{savenotes}
  \begin{table}
    \begin{tabularx}{\textwidth}{X p{.5\textwidth} p{.15\textwidth}}
      \toprule
      Header field key & Brief description & Value required \\
      \midrule
      \textsf{fileformat} & Fileformat and version & Yes \\
      \textsf{organism} & NCBI Taxonomic identifier\footnote{NCBI Taxonomy: a comprehensive update on curation, resources and tools, \url{https://doi.org/10.1093/database/baaa062}} & Yes \\
      \textsf{modification\textunderscore type} & A valid RNA type & Yes \\
      \textsf{modification\textunderscore names} & name:short\textunderscore name:primary\textunderscore base & Yes \\
      \textsf{assembly} & Genome assembly & Yes \\
      \textsf{annotation\textunderscore source} & Annotation source & Yes \\
      \textsf{annotation\textunderscore version} & Annotation version & Yes \\
      \textsf{sequencing\textunderscore platform} & Sequencing platform & No \\
      \textsf{basecalling} & Basecalling model information where relevant & No \\
      \textsf{bioinformatics\textunderscore workflow} & Link to bioinformatics workflow; program name, version, and/or call; information relevant to score, coverage, or frequency calculation; \textit{etc.} & No \\
      \textsf{experiment} & Information about or link to experimental protocol and design & No \\
      \textsf{external\textunderscore source} & Databank:ID of data & No \\
      \bottomrule
    \end{tabularx}
    \caption{\textbf{Header Fields.}}\label{tab:header}
  \end{table}
\end{savenotes}
 

\subsection{Data specification}\label{sec:data}

Each~\textbf{data line} contains 11~\textbf{\acs{bedRMod} field}s delimited by a~\textbf{field separator} (tab).
All~\textbf{fields} are mandatory~(\autoref{tab:fields}). Missing data is not allowed.
Additional optional~\textbf{field}s can be added, following the first 11~\textbf{field}s, according to the \acs{BED} specifications, but 
it is not recommended to use \acs{bedRMod} with exactly 12~\textbf{field}s, \textit{cf.}~\autoref{sec:custom_fields}.

\begin{savenotes}
  \begin{table}[H]
    \begin{adjustwidth}{-0.5in}{-0.5in}
      \begin{tabularx}{\linewidth}{r l l l L}
        \toprule
        Col & \acs{bedRMod} field & Type & Regex or range & Brief description \\
        \midrule
        1
        & \textsf{chrom}
        & String
        & \texttt{[[:alnum:]\_]\{1,255\}}\footnote{\texttt{[[:alnum:]\_]} is equivalent to the \ac{regex}~\texttt{[A-Za-z0-9\_]}. % chktex 8
        It is also equivalent to the Perl extension~\texttt{[[:word:]]}}
        & Chromosome name \\
        2 & \textsf{chromStart} & Int & $[0, 2^{64}-1]$ & \textbf{Feature} start position \\
        3 & \textsf{chromEnd} & Int & $[0, 2^{64} -1]$ & \textbf{Feature} end position \\
        4 
        & \textsf{name} 
        & String 
        & \texttt{[{\textbackslash}x20-{\textbackslash}x7e]\{1,255\}} 
        & \textbf{Feature} name and additional attributes \\
        5 & \textsf{score} & String & \texttt{[{\textbackslash}x20-{\textbackslash}x7e]\{1,255\}} & \textbf{Feature} confidence \\
        6 & \textsf{strand} & String & \texttt{[-+.]} & \textbf{Feature} strand \\
        7 & \textsf{thickStart} & Int & $[0, 2^{64}-1]$ & Thick start position, typically same as~\textsf{chromStart} \\
        8 & \textsf{thickEnd} & Int & $[0, 2^{64}-1]$ & Thick end position, typically same as~\textsf{chromEnd} \\
        9 & \textsf{itemRgb} & Int,Int,Int & \texttt{(}$[0, 255], [0,255], [0,255]$\texttt{) | 0} & Display color \\ % chktex 9
        10 & \textsf{coverage} & Int &  $(0, 2^{64}-1]$ & \textbf{Feature} coverage \\
        11 & \textsf{frequency} & Float & $[0, 100]$ & \textbf{Feature} frequency, \textit{i.e.} percentage of modification \\
        \bottomrule
      \end{tabularx}
    \end{adjustwidth}
    \caption{\textbf{\acs{bedRMod} Fields.}}\label{tab:fields}
  \end{table}
\end{savenotes}

In a \ac{bedRMod}~\textbf{file}, each~\textbf{data line} must have the same number of~\textbf{field}s.
The positions in~\textbf{\acs{bedRMod} field}s are all described in the~\textbf{0-based, half-open coordinate system}, exactly as 
described in the official \ac{BED} specifications.

\subsection{Coordinates}
Refer to the official \ac{BED} specifications.

\subsection{Simple attributes}
\begin{enumerate}
\item \textsf{name}: String that describes the~\textbf{feature}, \textit{i.e.} the modification. \textsf{name} must describe 
the modification using the \emph{short name} of the MODOMICS nomenclature\footnote{MODOMICS, \url{https://www.genesilico.pl/modomics/modifications}}, or the base modification code described in the SAMtags\footnote{SAM tags, \url{https://samtools.github.io/hts-specs}}, or a numeric ChEBI code\footnote{Chemical Entities of Biological Interest, \url{https://www.ebi.ac.uk/chebi}}. The MODOMICS \emph{short name} corresponding to~\textsf{name} is always described in the~\textbf{header field}~\textsf{modification\textunderscore names}. Additional name attributes are allowed, and must be comma-separated \textit{e.g.}~\texttt{a,DRACH,2}.

\item \textsf{score}: String representation of the confidence in calling this modification. Any measure of confidence is valid, but a \ac{bedRMod}~\textbf{file} with non-integer-like~\textsf{score} values outside the range $[0, 1000]$ may fail from being correctly displayed in a visual representation\footnote{\textit{cf.} bedtools definition of score, \url{https://bedtools.readthedocs.io/en/latest/content/general-usage.html?highlight=bed\%20format}}.

\item \textsf{coverage}: Integer between~0 and the maximum size of an unsigned 64-bit integer, excluding~0, representing the number of reads with a base aligned to this reference position for which this~\textbf{feature} is a modification. The primary or canonical base must be inferred by the modification, \textit{e.g.} for ~\texttt{m6A}, this is the number of reads with an~\texttt{A} aligned to this position.

\item \textsf{frequency}: Float\footnote{Decimal string representation of 64-bit floating point number, IEEE Standard for Binary Floating-Point Arithmetic. IEEE 754–1985, 1985.} between~0 and~100, including~0, representing the modification frequency, or stoichiometry. This can be the percentage of modified reads, or the ratio of the number of calls passing filters that were classified as a residue with the base modification reported for this~\textbf{feature} to the \emph{valid coverage}, multiplied by~100. See~\autoref{sec:fields} for an explanation of \emph{valid coverage}. A~\textsf{frequency} of~0 means that there is evidence that a given site is not modified, \textit{i.e.} the primary or canonical base is reported with a confidence quantified by a~\textsf{score}~$>$0.
\end{enumerate}

\subsection{Display attributes}
\begin{enumerate}
  \setcounter{enumi}{4}

\item \textsf{thickStart}: Included for compatibility, typically same as~\textsf{chromStart}.

\item \textsf{thickEnd}: Included for compatibility, typically same as~\textsf{chromEnd}.

\item \textsf{itemRgb}: Included for compatibility, typically~\texttt{0,0,0}.

\end{enumerate}

\subsection{Custom fields}\label{sec:custom_fields}

\textbf{Custom field}s defined by the~\textbf{file} creator may contain any printable 7-bit US \ac{ASCII} character (which includes spaces, but excludes tabs, newlines, and other control characters), as defined by the \ac{BED} format definitions.

A \acs{bedRMod}~\textbf{file} with exactly 12~\textbf{field}s, \textit{i.e.} containing one additional optional~\textbf{field}, may be implicitely 
assumed to be a~\textbf{BED12}~\textbf{file} by certain software and genome browsers, which can result in unexpected behaviour! 

\section{Examples}

\subsection[title]{Example bedRMod file from the \acs{bedRMod} and related specifications\footnote{\url{https://github.com/dieterich-lab/euf-specs/examples/bedrmod/example.bedrmod}}}\label{sec:example-bedrmod}

\begin{verbatim}
#fileformat=bedRModv2
#organism=9606
#modification_type=RNA
#modification_names=20607:m5C:C,21891:m6A:A
#assembly=GRCh38
#annotation_source=Ensembl
#annotation_version=93
#sequencing_platform=Illumina NovaSeq 6000
#basecalling=
#bioinformatics_workflow=workflow:https://github.com/XXX
#experiment=https://doi.org/10.XXX
#external_source=SRA:PRJNAXXXXXX,GEO:GSEXXXXXX
#chrom chromStart chromEnd name score strand thickStart thickEnd itemRgb coverage frequency
1 1391918 1391919 20607 20 - 1391918 1391919 0,0,0 42 42.56
2 8878712 8878713 20607 150 - 8878712 8878713 0,0,0 318 44.23
3 11980442 11980443 21891 78 + 11980442 11980443 0,0,0 111 56.20
4 17054111 17054112 20607 10 - 17054111 17054112 0,0,0 40 34.03
\end{verbatim}

\section{Recommended practice for the \acs{bedRMod} format}

\subsection{\acs{bedRMod} extension}\label{sec:recommend-ext}
The~\textbf{file} extension is~\texttt{.bedrmod},~\texttt{.bedmethyl}, or~\texttt{.bed}. Since the \ac{BED} format prohibits~\textbf{BED11}, there should be little confusion in general, but when~\textbf{custom field}s are defined, it is recommended to use the~\texttt{.bedrmod} or~\texttt{.bedmethyl} extension.

\subsection{Mandatory \acs{bedRMod} header fields}\label{sec:recommend-header}
These~\textbf{header field}s are not free text, and must conform to a controlled vocabulary.
\begin{itemize}
\item \textsf{fileformat}: A valid version of this specification, including the format name, \textit{e.g.}~\texttt{bedRModv2}. 

\item \textsf{modification\textunderscore type}: A valid RNA type\footnote{For example RNA or mRNA, tRNA, or rRNA. A next version of this specification should prescribe a controlled vocabulary by providing a reference to an established RNA ontology.}.

\item \textsf{modification\textunderscore names}: A comma-separated dictionary mapping in the form name:short\textunderscore name:primary\textunderscore base, where name corresponds to the~\textsf{name}~\textbf{field} of a~\textbf{data line}, short\textunderscore name to the \emph{short name} of the MODOMICS nomenclature, and primary\textunderscore base is the canonical or primary sequence base, \textit{e.g.}~\texttt{21891:m6A:A}. A value is also required when using MODOMICS \emph{short names}, \textit{e.g.}~\texttt{m6A:m6A:A}. All modifications present in the~\textbf{file} must be included, in a comma-separated list of items.

\item \textsf{organism}: A valid NCBI Taxonomic identifier, \textit{e.g.}~\texttt{9606}.

\item \textsf{assembly}: The name of a valid assembly, \textit{e.g.} using the Ensembl terminology,~\texttt{GRCh38}. 

\item \textsf{annotation\textunderscore source}: The name of a valid annotation, \textit{e.g.}~\texttt{Ensembl}. 

\item \textsf{annotation\textunderscore version}: A valid version for the annotation source, \textit{e.g.}~\texttt{110}. 
\end{itemize}

\subsection{\acs{bedRMod} header fields}\label{sec:recommend-fields}
These~\textbf{header field}s can be left without a value, but the key must always be present. The value is free text, although it is strongly recommended
to reference established \textsf{sequencing\textunderscore platform}s, \textsf{basecalling} models, \textsf{bioinformatics\textunderscore workflow}s, or \textsf{external\textunderscore source}s using an exact terminology and/or recognized identifiers.  
\begin{itemize}
\item \textsf{sequencing\textunderscore platform}: Typically, the name of the sequencing instrument or device, including key specifications if relevant, \textit{e.g.}~\texttt{ONT MinION}.

\item \textsf{basecalling}: Basecalling model such as name of versioned model, reference to published model, and/or additional details on training, \textit{e.g.}~\texttt{dna\textunderscore r9.4.1\textunderscore e8\textunderscore sup$@$v3.6}.

\item \textsf{bioinformatics\textunderscore workflow}: Program name, version, and/or call used to generate the~\textbf{file}, or link to open source bioinformatics workflow, including version and/or any additional details to facilitate data lineage. The information should be sufficient to reproduce the content of the~\textbf{file}.

\item \textsf{experiment}: Supplementary information about experimental protocol, design, or the content of the~\textbf{file} such as conditions used, number of replicates, \textit{etc.}, or link to an openAIRE repository.

\item \textsf{external\textunderscore source}: A comma-separated list of sources of the form Databank:ID, \textit{e.g.}~\texttt{GEO:GSEXXXXXX, Zenodo:10.XXX/zenodo.XXXXXXXX}. Free text is allowed, but this specification recommends using the format described here. Together with~\textsf{bioinformatics\textunderscore workflow}, this should allow to reproduce the content of the~\textbf{file}.
\end{itemize}


\subsection{\acs{bedRMod} fields}\label{sec:fields}
\begin{itemize}
\item \textsf{chrom}: The name of each chromosome should match the names from a reference genome assembly, as given in the~\textbf{header}. For example, if~\texttt{\#assembly=GRCh38}, then chromosomes should be named~\texttt{1} to~\texttt{22}, \texttt{X}, \texttt{Y}, and~\texttt{MT}, consistently through the~\textbf{file}.

\item \textsf{name}: The MODOMICS \emph{short names} or the ChEBI codes should be used in preference to the base modification codes described in the SAMtags.

\item \textsf{score}: The \emph{valid coverage} should be used as a measure of confidence. This can be \textit{e.g.} the number of calls passing filters (classified as modified and unmodified) at the reported modification position\footnote{\textit{cf.} \url{https://nanoporetech.github.io/modkit}}, or the number or reads remaining after filtering and used to infer the modificaton status.
\end{itemize}

\subsection{Whitespace}\label{sec:whitespace}
We recommend that only a single tab~(\texttt{{\textbackslash}t}) be used as~\textbf{field separator}, \textit{cf.} offical \ac{BED} specifications.

\section{Information supplied out-of-band}

A \ac{bedRMod} \textbf{file} contains 11 required~\textbf{field}s, any additional~\textbf{field}s may require information that must be supplied out-of-band. A common practice is to include a~\textbf{comment line} after the~\textbf{header} to describe the~\textbf{field}s used in the~\textbf{file}, \textit{cf.}~\autoref{sec:example-bedrmod}.

The semantics of~\textbf{field}s such as~\textsf{score},~\textsf{coverage}, and~\textsf{frequency} can be included in the~\textsf{header} using the \textbf{header field}~\texttt{bioinformatics\textunderscore workflow}.

\section{Acronyms}

% using the optional argument to acronym to set the label width causes it to use the list environment instead of description, which means we can't set nosep easily
\setlist[description]{labelwidth=\widthof{\textbf{\acs{bedRMod}}},nosep}
\begin{acronym}
  \acro{ASCII}{American Standard Code for Information Interchange}
  \acro{BED}{Browser Extensible Data}
  \acro{bedRMod}{Browser Extensible Data for RNA modification}
  \acro{ChEBI}{Chemical Entities of Biological Interest}
  \acro{NCBI}{National Center for Biotechnology Information}
  \acro{regex}{regular expression}
\end{acronym}

\section{Acknowledgments}

We thank the \acf{bedRMod} format specification working group and the Modkit developers.

\end{document}

% chktex-file 17

%%% Local Variables:
%%% mode: latex
%%% TeX-master: t
%%% End:
